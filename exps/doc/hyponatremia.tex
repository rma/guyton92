\label{exp:hyponatremia}

This experiment is based on the scenario of overly-rapid correction of chronic hyponatremia, as presented by \citeauthor{Kamel2010} \cite{Kamel2010}.

\subsection{Key variables}
\begin{table}[!h]
  \centering
  \begin{tabularx}{\textwidth}{lXrr}
    \toprule
    Variable & Description & Units (1972) & Units (1992) \\
    \midrule
    VUD & Urinary output & mL/min & \textbf{L/min} \\
    PA & Mean arterial pressure & mmHg & mmHg \\
    AUP & Sympathetic stimulation & (ratio) & \textbf{Missing} \\
    QLO & Cardiac output & L/min & L/min \\
    BFM & Muscle blood flow & L/min & L/min \\
    MMO & Oxygen usage by muscle cells & mol/min & \textbf{L/min} \\
    \bottomrule
  \end{tabularx}
  \caption{Key variables for the experiment. Note that the units of VUD have changed in the 1992 version of the model. This must be taken into account when comparing the model output to the data in \autoref{tbl:hyponatremia}.}
\end{table}

\subsection{Simulation parameters}
\begin{enumerate}
  \item After x units, blah.
  \item Total simulation time: 9 minutes (540 seconds).
\end{enumerate}

\subsection{General observations}
